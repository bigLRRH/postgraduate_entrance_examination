\documentclass{article}
\usepackage{amsmath}
\usepackage{amsfonts}

\begin{document}

\title{CPU执行时间计算}
\author{}
\date{}
\maketitle

\section*{公式}
\begin{align*}
    \text{CPU执行时间} &= \frac{\text{程序总时钟周期数}}{\text{时钟频率}} \\
                        &= \text{程序总时钟周期数} \times \text{时钟周期}
\end{align*}

\begin{equation}
    \text{程序总时钟周期数} = \text{程序总指令条数} \times \text{CPI}
\end{equation}

\begin{equation}
    \text{程序总时钟周期数} = \sum_{i=1}^{n} (\text{CPI}_i \times C_i)
\end{equation}

其中,第$i$种指令的条数和CPI分别为$C_i$和$\text{CPI}_i$。

\begin{equation}
    \text{CPI} = \sum_{i=1}^{n} (\text{CPI}_i \times F_i) = \frac{\text{程序总时钟周期数}}{\text{程序总指令条数}}
\end{equation}

\section*{示例计算}
假设一个程序包含三种不同的指令,其指令条数和相应的CPI如下:
\begin{align*}
    C_1 &= 1000, \quad \text{CPI}_1 = 2 \\
    C_2 &= 2000, \quad \text{CPI}_2 = 3 \\
    C_3 &= 3000, \quad \text{CPI}_3 = 1
\end{align*}

计算总指令条数:
\begin{equation}
    C_{\text{total}} = 1000 + 2000 + 3000 = 6000
\end{equation}

计算程序总时钟周期数:
\begin{equation}
    \text{总时钟周期数} = 1000 \times 2 + 2000 \times 3 + 3000 \times 1 = 2000 + 6000 + 3000 = 11000
\end{equation}

计算整体CPI:
\begin{equation}
    \text{CPI} = \frac{11000}{6000} = 1.83
\end{equation}

如果时钟频率为3 GHz(即时钟周期为 $\frac{1}{3 \times 10^9}$ 秒),则CPU执行时间为:
\begin{equation}
    \text{CPU执行时间} = 11000 \times \frac{1}{3 \times 10^9} \approx 3.67 \times 10^{-6} \text{秒} = 3.67 \text{微秒}
\end{equation}

\end{document}